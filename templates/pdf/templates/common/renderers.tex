%! Author = lps
%! Date = 06/01/2021
%! Renderer macros render given data in a specific form

%- macro render_error(message)
    \todo[inline]{"Fix this"}
    \begin{lstlisting}[language=PostScript, caption=Error message]
            \VAR{ message|replace("\n","
") }
    \end{lstlisting}
%- endmacro

%# The default macro for showing/wrapping the fetch results
%- macro render_fetch_results(content, error)
    %- if error
        \VAR{ render_error(error) }
    %- else
        %- if content is undefined
            \VAR{ render_error("Some content expected but none was found.") }
            \todo{"Fix this"}
        %- else
            \VAR{ caller() }
        %- endif
    %- endif
%- endmacro

%- macro render_section_header(instance_uri, instance_label, level, is_concept=false)
    %- set super_script_symbol = "Cn" if is_concept else "Cl"
    %- set text = instance_label
    %- if level == 1
\part{\VAR{ text } }
\label{sec:\VAR{instance_uri}}
    %- elif level == 2
\chapter{ \VAR{ text } \textsuperscript{\normalsize \VAR{super_script_symbol} } }
\label{sec:\VAR{instance_uri}}
    %- elif level == 3
\section{\VAR{ text } \textsuperscript{\normalsize \VAR{super_script_symbol} } }
\label{sec:\VAR{instance_uri}}
    %- elif level == 4
\subsection{\VAR{ text} \textsuperscript{\normalsize \VAR{super_script_symbol} } }
\label{sec:\VAR{instance_uri}}
    %- elif level == 5
\subsubsection{\VAR{ text } \textsuperscript{\normalsize \VAR{super_script_symbol} } }
\label{sec:\VAR{instance_uri}}
    %- else
\section{Headding level out of boundaries: use a number 1 -- 5}
    %- endif
%- endmacro

%- macro render_description_item_plain_literal(result_set, property_id, property_label, is_integer=False, uri_to_qname=False)
    %- if not result_set.empty and property_id in result_set.columns
        %- set values = result_set[property_id].dropna()
        %- if is_integer
            %- set values = values.astype("int64", copy=False, errors="ignore")
        %- endif
        %- if uri_to_qname
            %- set values = values.apply(namespaces.uri_to_qname)
        %- endif
        %- if not values.empty
            \paragraph*{\VAR{ property_label } }%\mbox{}\\
                %- for idx, item in values.iteritems()
\noindent \VAR{ item|escape_latex|replace("\n","
") } \mbox{}\\
                %- endfor
        %- endif
    %- else
        \VAR{ render_error("Either the result set is empty or the property id "~property_id~" doesn't exist") }
    %- endif
%- endmacro

%- macro render_description_item_local_iri(result_set, property_id, property_label, link_label_property_id=None)
    %- if not result_set.empty and property_id in result_set.columns
        %- set values =  result_set[property_id].dropna()
        %- set values = values.apply(namespaces.uri_to_qname)
        %- if not values.empty
            \paragraph*{\VAR{ property_label } } %\mbox{}\\
            %- for idx, item in values.iteritems()
                %- if link_label_property_id is none
\noindent\hyperref[sec:\VAR{ item }]{ \VAR{ item|escape_latex } } \mbox{}\\
                %- else
\noindent\hyperref[{sec:\VAR{ item }]{ \VAR{ link_label_property_id|escape_latex } } \mbox{}\\
                %- endif
            %- endfor
        %- endif
    %- else
        \VAR{ render_error("Either the result set is empty or the property id "~property_id~" doesn't exist") }
    %- endif
%- endmacro


%- macro render_description_item_listing(result_set, property_id, property_label, listing_language=XML)
    %- if not result_set.empty and property_id in result_set.columns
        %- set values = result_set[property_id].dropna()
        %- if not values.empty
\paragraph*{\VAR{ property_label } } %\mbox{}\\
                %- for idx, item in values.iteritems()
\begin{lstlisting}[language=\VAR{listing_language}]
\VAR{ item | trim }
\end{lstlisting} \mbox{}\\
                %- endfor
        %- endif
    %- else
        \VAR{ render_error("Either the result set is empty or the property id "~property_id~" doesn't exist") }
    %- endif
%- endmacro

%- macro render_inline_values(result_set, property_id, label, default_value=None, separator=',', is_integer=False, uri_to_qname=False)
    %- if not result_set.empty and property_id in result_set.columns
        %- set values = result_set[property_id].dropna()
        %- if is_integer
            %- set values = values.astype("int64", copy=False, errors="ignore")
        %- endif
        %- if not values.empty
            %- if uri_to_qname
                %- set values = values.apply(namespaces.uri_to_qname)
            %- endif
\VAR{ label + " " + ", ".join(values.astype('str').tolist()) |escape_latex|replace("\n","
") }
        %- elif default_value
\VAR{ label + " " + default_value |escape_latex|replace("\n","
") }
        %- endif
    %- else
        \VAR{ render_error("Either the result set is empty or the property id "~property_id~" doesn't exist") }
    %- endif
%- endmacro
