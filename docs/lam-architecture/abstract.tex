\section*{Abstract}

	Legal Analysis Methodology (LAM) for OP legal data aims to define the semantic aspects of the Publications Office of the European Union (OP) legal data on very specific level. It provides description of the metadata elements meaning, links them to various document types published in Official Journal or on EUR-Lex and describes the rules for attribution of values. It serves as an overarching framework for describing usage of a suite of legal data standards as applied in in their working context. 
	
	LAM plays an important role in the discovery of the EU legal resources (CELLAR, EUR-Lex, OP Portal), which is a central objective for the OP at large. The proper use of LAM leads to a significant decrease in the missing, confusing, incorrect or insufficient data for the stakeholders. In addition, it can increase the data interoperability and can facilitate automation for legal data at various levels.

    This document provides a working definition of the architectural stance and design decisions that are to be adopted for the LAM data maintenance and dissemination life-cycle and the supporting services. This process is aligned with the semantic asset publication workflow currently employed by the Metadata and Standardisation Unit at the Publications Office of the European Union (OP).