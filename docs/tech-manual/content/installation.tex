\section{Requirements}
\label{sec:requirements}

Although Docker can be executed on any platform, for performance and security reasons we recommend using a Linux OS with kernel version 5.4.x or higher. The services have been tested on Ubuntu 20 server.

There is a range of ports that must be available on the host machine as they will be bound to by different docker services. Although the system administrator may choose to change them by changing the values in of specific environment variables. The inventory of pre-configured ports is provided in Table \ref{tab:port-inventory}.

\begin{longtable}[c]{@{}p{3.64cm}p{1.25cm}p{1.25cm}p{1.9cm}p{5cm}@{}}
	\toprule
	Service name  & HTTP port UI & HTTP port API & Mounted volume \\* \midrule
	\endfirsthead
	%
	\multicolumn{5}{c}%
	{{\bfseries Table \thetable\ continued from previous page}}              \\
	\endhead
	%
	\bottomrule
	\endfoot
	%
	\endlastfoot
	%
	LAM validator & 10002       & 10001          &     rdf-validator -shacl-shape          \\* \hline
	LAM Generation Service & 8050         & 4050               &                \\* \hline
	LAM dedicated triple store & 3010                 &          &    lam-fuseki            \\* \hline
	\caption{Port usage inventory}
	\label{tab:port-inventory}                                               \\
\end{longtable}

%	\vfil
The minimal hardware requirements are as follows
\begin{enumerate}
	\item CPU: Dual core 3Ghz
	\item RAM: 8Gb
	\item SDD system: 2Gb
	\item SDD data: 8Gb 
\end{enumerate}

\section{Installation}
\label{sec:installation}

In order to run the services it is necessary to have Docker \citep{docker} service and docker-compose tool installed. To install them follow the instructions provided on the official websites

\begin{enumerate}
	\item Docker - \url{https://docs.docker.com/engine/install}
	\item Docker Compose - \url{https://docs.docker.com/compose/install}
\end{enumerate}

In case you are using Debian-like OS such as Ubuntu, you may simply run the following Bash commands to install and set the appropriate permissions.

\begin{lstlisting}[language=bash,]
sudo apt -y install docker.io docker-compose git make
sudo groupadd docker
sudo usermod -aG docker $USER
newgrp docker
\end{lstlisting}

Please note that the \textit{docker.io} package si installed rather than the \textit{docker-ce} one.

Next, to launch the services, clone the Git repository or unzip the project containing the \textit{docker-compose.yml}, \textit{.env} file and the \textit{Makefile}. 

\begin{lstlisting}[language=bash,]
# example of cloning the project
git clone https://github.com/meaningfy-ws/lam-workflow.git
cd lam-workflow
\end{lstlisting}

You may choose to adjust these files as necessary on your system. Then change directory into the \textit{lam-workflow} folder and Makefile commands to start and stop services will be available. To start services run

\begin{lstlisting}[language=bash,]
make start-services
\end{lstlisting}

To stop the services run 

\begin{lstlisting}[language=bash,]
make stop-services
\end{lstlisting}

Downloading the Docker images will be triggered automatically on first request to start the services.

To start the services using Makefile

\begin{lstlisting}[language=bash,]
make location=</your-custom/shapes/location> validator-set-shacl-shapes
make start-services
\end{lstlisting}

To stop the services using Makefile

\begin{lstlisting}[language=bash,]
make stop-services
\end{lstlisting}

To start services without Makefile first prepare the volume with LinkedPipes ETL configurations file like this

\begin{lstlisting}[language=bash,]
docker rm temp | true
docker volume rm rdf-validator-shacl-shapes | true
docker volume create rdf-validator-shacl-shapes
docker container create --name temp -v rdf-validator-shacl-shapes:/data busybox
docker cp <your-custom/shapes/location>. temp:/data
docker rm temp
\end{lstlisting}

then start the services

\begin{lstlisting}[language=bash,]
docker-compose --file docker/docker-compose.yml --env-file docker/.env up -d
\end{lstlisting}

To stop the services run

\begin{lstlisting}[language=bash,]
docker-compose --file docker/docker-compose.yml --env-file docker/.env down
\end{lstlisting}

The detailed explanation on how to configure them is provided in the Configuration section for each of these services (See Section \ref{sec:rdf-validator-ss}).
